\documentclass{instrukcja}
\usepackage[polish]{babel}
\usepackage[utf8]{inputenc}
\usepackage[OT4]{fontenc}

\begin{document}

\materialnumber{1}
\course[Info I]{Informatyka I}
\material[Lab 1]{Instrukcja 1}
\author{\L{}. \L{}aniewski-Wo\l{}\l{}k}
\materialtitle

\section*{Wstęp}

Rozpakuj projekt. Otwórz plik {\tt projekt.sln}. W otwartym projekcie są następujące pliki:
\begin{enumerate}
\item {\tt main.cpp} --- główny plik z kodem. Tu piszemy nasz program
\item {\tt winbgi2.cpp} --- plik z funkcjami graficznymi
\item {\tt winbgi2.h} --- plik z definicjami funkcji graficznych
\end{enumerate}

{\bf{\red Pamiętaj:} Często kompiluj projekt i patrz, czy wszystko działa!}

\section{Pierwsze kreski}

Wewnątrz funkcji {\tt main} wpisz:
\begin{verbatim}
graphics( 200, 200);
line( 0, 0, 200, 200);
line( 200, 200, 0, 0);
wait();
\end{verbatim}
{\bf{\red Uwaga:} Zawsze pamiętaj o średnikach!}

Skompiluj i uruchom projekt. Pierwsza linia tworzy okno grafiki, dwie następne rysują linie, zaś ostatnia czeka z zamknięciem okna na naciśnięcie dowolnego klawisza.

\subsection*{Ćwiczenia}
Używając funkcji {\tt line(x1,y1,x2,y2)} i {\tt circle(x,y,r)}, wykonaj następujące zadania:
\begin{itemize}

\item Zidentyfikuj, jak ułożony jest układ współrzędnych (X,Y) w oknie.
\item Narysuj kwadrat.
\item Narysuj ludzika.
\item Narysuj koła olimpijskie.
\end{itemize}

\section{Zmienne}
Pewne powtarzające się parametry (jak pozycję, promień, itp), możemy zastąpić zmiennymi. Następnie z nich wyliczyć odpowiednie współrzędne np:
\begin{verbatim}
int r,h;
h = 100;
r = 50;
line( 10, 0, 0, h);
line( 10, 0, 2*r, h);
circle( 10+r, h, r);
\end{verbatim}
Możemy używać wszelkich działań i funkcji matematycznych: {\tt +, -, *, /, sin(), \ldots}.

{\bf {\red Pamiętaj:} Pierwsza linia deklaruje zmienne. Trzeba zadeklarować wszystkie zmienne, których będziesz używać! (szczegóły, na kolejnych zajęciach).}

Zauważ, że wartość zmiennej jest nadpisywana, więc możemy napisać:
\begin{verbatim}
int w;
w = 50;
circle( 10, w, 10);
w = w + 20;
circle( 10, w, 10);
w = w + 20;
circle( 10, w, 10);
w = w + 20;
circle( 10, w, 10);
\end{verbatim}
W efekcie wyświetlą się cztery kółka narysowane koło siebie. Przetestuj.
\subsection*{Ćwiczenia}
Każdy program przetestuj dla paru ustawień zmiennych, by zobaczyć czy działa poprawnie.
\begin{itemize}
\item Napisz program, który dla zmiennych x,y,s, tworzy okno o rozmiarach x,y i na środku narysuje koło o promieniu s.
\item Dla zmiennej {\tt d}, narysuj cztery dotykające się koła o średnicy d w prawym górnym rogu okna.
\item Dla zmiennej {\tt y} narysuje koła olimpijskie w odległości {\tt y} od górnej krawędzi.
\item Skopiuj poprzedni kod trzy razy i w każdym fragmencie zmodyfikuj wartość zmiennej {\tt y}.
\end{itemize}

\section{Pętle}
Pierwszą automatyzacją są pętle. Pętla wykonuje pewną operację, dopóki pewien warunek jest spełniony. Np:
\begin{verbatim}
int x;
x = 0;
while (x < 200) {
    line(x,10,x,190);
    x = x + 10;
}
\end{verbatim}
Taki program będzie wykonywany w następujący sposób:
\begin{itemize}
\item wpisujemy $0$ do zmiennej $x$
\item sprawdzamy, czy $x<200$
\item rysujemy linię
\item zwiększamy zmienną $x$ o $10$
\item i znów: sprawdzamy, czy $x<200$
\item rysujemy linię
\item zwiększamy zmienną $x$ o $10$
\item sprawdzamy, czy $x<200$
\item rysujemy linię
\item \ldots
\item gdy wreszcie $x$ przekroczy $200$, pętla się skończy i program pójdzie dalej.
\end{itemize}

Ostatecznie program narysuje pionowe kreski dla kolejnych {\tt x} = 0, 10, 20, \ldots.

{\bf {\red Zauważ:} Program nie narysuje linii dla $x=200$, bo komputer najpierw sprawdzi, że $x\not< 200$ i przerwie pętlę.}


\subsection*{Ćwiczenia}
\begin{itemize}
\item Napisz program, który narysuje kratkę z odstępem {\tt w}
\item Narysuj rząd stycznych do siebie kół o promieniu $r$, zaczynając od lewej strony. Przemyśl: jeśli $x$ to pozycja środka koła, to jaka wartość powinna jej być przypisana przed pętlą, o ile powinna być zwiększana i jaki warunek musi spełniać, by nie rysować poza oknem?!
\item Pisząc jedną pętlę w drugiej, zapełnij cały obrazek przylegającymi kółkami.
\item[*] Czy da się je lepiej upakować?
\item Narysuj rząd kółek, których promienie zmniejszają się jak $\frac{1}{n}$.
\end{itemize}

\end{document}
