\documentclass{instrukcja}        
\usepackage[polish]{babel}
\usepackage[utf8]{inputenc}
\usepackage[OT4]{fontenc}

\begin{document}

\materialnumber{6}
\course[Info I]{Informatyka I}
\material[Lab 6]{Instrukcja 6}
\author{B. Górecki}
\materialtitle


\section{Współpraca z plikami}
Praktyczny program (szczególnie inżynierski) bardzo często musi współpracować z plikami. Czasem, przy obliczeniach trwających wiele godzin lub dni wręcz zależy nam na tym, by program działał samodzielnie bez potrzeby interakcji ze strony użytkownika. Wróćmy jednak do plików. Najczęściej chodzi o możliwość wczytania danych wejściowych z jednego (bądź wielu) plików, przeprowadzenie obliczeń wewnątrz programu i zapisanie wyników do innego pliku (bądź wielu plików).\\
W języku C komunikacja z plikami prowadzona jest niemalże identycznie, jak czytanie danych z klawiatury i wydruk na ekran, co realizowaliśmy za pomocą znanych już funkcji {\tt scanf} oraz {\tt printf}. Musimy jednak najpierw określić, jaki plik chcemy utworzyć bądź otworzyć i w jakim celu go tworzymy/otwieramy. Ponadto, analogiem instrukcji {\tt printf} do zapisu do plików jest instrukcja {\tt fprintf}, zaś funkcja {\tt scanf} jest zastąpiona przez funkcję {\tt fscanf}. Przyjrzyjmy się przykładowemu kodowi źródłowemu.
\begin{verbatim}
FILE *f; // deklarujemy zmienną typu FILE o nazwie f
         // tak naprawde to wskaznik (o tym jednak w pozniej)

f = fopen("plik.txt", "w"); // otwieramy plik o nazwie plik.txt
                            // z zamiarem zapisu ("w") i przypi-
                            // sujemy go do zmiennej o nazwie f

fprintf(f, "Zapisujemy wlasnie ten tekst do pliku\n");

fclose(f); // zamykamy plik
\end{verbatim}
Warto zwrócić uwagę na trzy kwestie. Po pierwsze, w funkcji {\tt fprintf} (to samo dotyczy funkcji {\tt fscanf} jako pierwszy argument trzeba podać {\it strumień} - de facto nazwę zmiennej typu {\tt FILE}, z którym zachodzi komunikacja (zapis lub odczyt). Dlatego tu podajemy {\it f}, bo tak właśnie nazwaliśmy naszą zmienną. Druga uwaga: Powyższy zapis można nieco skompresować (połączyć deklarację zmiennej typu FILE z otwarciem pliku). Ponadto, istotne jest, by sprawdzić, czy plik udało się otworzyć. W przeciwnym razie jakiekolwiek operacje nie miałyby sensu lub skończyły się błędem naszego programu. Zmodyfikujmy więc nasze instrukcje. Teraz wyglądają tak:
\begin{verbatim}
FILE *f = fopen("plik.txt", "w")

if(f == NULL)
{
   printf("Blad otwarcia pliku\n");
   exit(-1); // zakonczenie programu
}

// Tu wykonujemy operacje na pliku (w naszym przypadku zapis)
// Gdy plik juz nie bedzie wiecej potrzebny w naszym programie,
// koniecznie go zamykamy!

fclose(f);
\end{verbatim}
Pliki można otworzyć nie tylko w trybie zapisu {\it (ang. write)} {\tt w} (który zawsze czyści plik i wypełnia go od nowa), ale również w trybie dopisywania do pliku {\it (ang. append)} {\tt a} lub czytania z pliku {\it (ang. read)} {\tt r}. Można również wybrać, czy tworzony/czytany plik ma być obsługiwany w trybie tekstowym czy binarnym. Służą do tego odpowiednio sekwencje {\tt t} i {\tt b}. Przykładowe instrukcje zaprezentowano poniżej. Zauważmy też, że można otworzyć w tym samym czasie kilka plików.
\begin{verbatim}
void main()
{
   int a = 3;

   FILE *f = fopen("plik1.txt", "wt"); // Zapis w trybie tekstowym
   FILE *g = fopen("plik2.dat", "wb"); // Zapis w trybie binarnym
   FILE *InnyPlik = fopen("Dane.txt", "r"); // Czytanie z pliku

   if((f == NULL) || (g == NULL) || (InnyPlik == NULL))
   {
      printf("Nie udalo sie otwarcie choc jednego z plikow\n");
      exit(-1);
   }

   fprintf(f, "Zapisujemy wartosc a do plik1.txt, a = %d\n", a);
   fprintf(g, "Binarnie zapisujemy ten tekst do plik2.dat\n");
   fscanf(InnyPlik, "%d", &a); // Wczytujemy z pliku Dane.txt
// liczbe calkowita i przypisujemy jej wartosc do zmiennej a

   // Tu mozemy wykonac jeszcze inne operacje na otwartych plikach
   
   fclose(f);
   fclose(g);
   fclose(InnyPlik);
}
\end{verbatim}
W powyższym przykładzie zaprezentowaliśmy jednocześnie użycie funkcji {\tt fscanf}, która działa analogicznie do dobrze już znanej funkcji {\tt scanf}.
\subsection*{Uwaga}
Wszystkie funkcje związane z obsługą plików znajdują się w bibliotece {\tt stdlib.h}. W związku z tym do pliku programu należy dołączyć instrukcję preprocesora załączającą tę bibliotekę: {\tt \#include~<stdlib.h>}
\subsection*{Ćwiczenia}
W praktyce inżynierskiej pliki często zawierają dane pochodzące z eksperymentu lub symulacji. Plik {\tt przebieg.txt} zawiera fragment przebiegu czasowego wartości trzech składowych prędkości {\it (u, v, w)} pochodzących z symulacji przepływu powietrza przez dużą turbinę wiatrową. Chwilowe wartości tych składowych zostały zebrane z punktu znajdującego się tuż za turbiną. Napisz program, który:
\begin{itemize}
\item Otworzy plik.
\item Wczyta dane z pliku do trzech tablic {\tt u}, {\tt v}, {\tt w} zadeklarowanych statycznie (każda o rozmiarze 2000 - za tydzień będzie o lepszej metodzie deklaracji dużych tablic). Czytanie zrealizuj z użyciem pętli {\tt for}. Obejrzyj plik, aby przyjrzeć się, w jaki sposób ułożone są dane (każda z kolumn odpowiada jednej ze składowych prędkości {\it (u, v, w)}; kolejne wiersze odpowiadają kolejnym krokom czasowym).
\item Po wczytaniu wszystkich wartości do tablic obliczy średnią każdej ze składowych. Średnia wyrażona jest wzorem:
\begin{equation}
\bar{u} = \frac{\displaystyle{\sum_{i=1}^n u_i}}{n}
\end{equation}
\item Obliczy odchylenie standardowe dla każdej ze składowych. Odchylenie standardowe dane jest wzorem:
\begin{equation}
\displaystyle{\sigma = \sqrt{\frac{\displaystyle{\sum_{i=1}^n (u_i - \bar{u})^2}}{n-1}}}
\end{equation}
\item Zapisze do innego pliku raport z obliczeń, w którym poda wszystkie obliczone wielkości oraz wydrukuje to samo na ekran.
\item Wczytaj też plik {\tt przebieg.txt} do arkusza kalkulacyjnego i utwórz wykres obrazujący te przebiegi. Oceń krytycznie wyniki uzyskane swoim programem na podstawie obserwacji wykresu. Czy średnie i odchylenia standardowe mają wiarygodne wartości?
\end{itemize}
\subsection*{Wskazówka}
Zauważ, że każda suma daje się łatwo policzyć z użyciem pętli {\tt for} w następujący sposób:
\begin{verbatim}
double suma = 0;

for(int i = 0; i<n; i++)
{
   suma += a[i];
}
\end{verbatim}
To tylko wskazówka. Oczywiście musisz zmodyfikować powyższy kod tak, aby liczył sumy z powyższych wzorów.
\section{Ważne: Dalej o funkcjach}
Wiemy, że funkcje mogą przyjmować argumenty. Dowiedzieliśmy się też, że funkcje mogą zwracać wartości. Zmodyfikuj swój kod tak, aby odpowiednie bloki instrukcji były realizowane w funkcjach {\tt Srednia} i {\tt OdchylenieStandardowe}. Powinny mieć takie nagłówki:
\begin{verbatim}
double Srednia(double *tablica, int n);
double OdchStd(double *tablica, double WartoscSrednia, int n);
\end{verbatim}
Następnie zmodyfikuj kod funkcji {\tt main} tak, aby część dotycząca obliczeń dała się zwięźle zapisać w poniższej postaci:
\begin{verbatim}
void main()
{
   (...) // Deklaracje i kod wczytujacy dane

   um = Srednia(u, n);
   vm = Srednia(v, n);
   wm = Srednia(w, n);

   u_std = OdchStd(u, um, n);
   v_std = OdchStd(v, vm, n);
   w_std = OdchStd(w, wm, n);

   (...) // Dalsza czesc programu zajmujaca sie raportowaniem wynikow
}
\end{verbatim}
\section*{*Dla dociekliwych}
Obliczenia na komputerze prowadzone są ze skończoną dokładnością. Zmodyfikuj swój kod tak, aby bieżąca wartość średniej była liczona ,,w locie'' - w trakcie czytania danych z pliku (naturalnie będzie to średnia wartość przeczytanych dotąd elementów). Wystarczy, że zrobisz to dla jednej składowej prędkości (np. {\it u}). Możesz tę średnią też na bieżąco podczas czytania danych drukować na ekran. Na końcu porównaj wartość średniej uzyskanej w ten sposób z wartością policzoną {\it a posteriori} w poprzednim poleceniu.\\
Pseudokod algorytmu znajdziesz poniżej. Zapisz go w sposób zrozumiały dla komputera, w języku C.
\begin{verbatim}
biezaca_srednia = 0

Petla po i od 1 do n (czytajaca dane)
{
   PrzeczytajNowyElementZPlikuIWpiszGoDoTablicy

   biezaca_srednia = (biezaca_srednia*(i-1) + u[i])/i
}

// Po zakonczeniu petli biezaca_srednia to srednia z calego zbioru
\end{verbatim}
Zastanów się, dlaczego taki algorytm liczenia średniej w sensie matematycznym prowadzi do tak samo zdefiniowanej średniej. Jeśli trudno ci go zrozumieć, wymyśl sobie zbiór czteroelementowy i wykonaj go krok po kroku na kartce.
\subsection*{Pytanie}
Czy obie średnie (policzone na komputerze dwoma sposobami) mają tę samą wartość? Czy coś się zmieni, gdy weźmiesz inną składową prędkości?

\end{document}
